\documentclass[addpoints,12pt,answers]{exam}

\usepackage{color}
\usepackage{amsmath,amssymb}
\usepackage{listings}
\usepackage{CJK}

\newcommand{\tf}[1][{}]{%
	\fillin[#1][0.25in]%
}
\newcommand{\tabincell}[2]{\begin{tabular}{@{}#1@{}}#2\end{tabular}}

\checkboxchar{$\Box$}
\checkedchar{$\blacksquare$}
\CorrectChoiceEmphasis{\color{red}}

\pagestyle{headandfoot}
\runningheadrule
\firstpageheader{BI296}{Quiz 2}{March 31, 2017}
\runningheader{BI296}
			{Quiz 2, Page \thepage\ of \numpages}
			{March 31, 2017}
\firstpagefooter{}{}{}
\runningfooter{}{}{}

\begin{document}
\begin{CJK*}{UTF8}{gbsn}

\vspace{0.1in}

\makebox[\textwidth]{学号、姓名: \enspace\hrulefill}

\vspace{0.1in}

\makebox[\textwidth]{教师: \hrulefill}

\section*{一、选择题}

\begin{questions}

\question[1]
对一个已经存在的文件执行\texttt{touch}命令后,修改的时间戳是
\begin{choices}
	\choice atime
	\choice ctime
	\choice mtime
	\choice 以上皆是
\end{choices}
\answerline


\question[1]
下列命令中,能同时查看两个以上文件内容的是
\begin{choices}
	\choice cat, tac
	\choice less, more
	\choice head, tail
	\choice 以上皆是
\end{choices}
\answerline

\question[1]
用cp和ln都可以“复制”文件,两者的差别是什么?
\begin{choices}
	\choice 前者创建一个独立的文件;后者创建的文件与前者不独立
	\choice 前者不改变源文件的所有属性
	\choice cp产生的文件具有不同的inode;而ln产生的硬连接文件与源文件具有相同的inode 
	\choice ln产生的符号连接文件与其他文件类型完全不同
\end{choices}
\answerline

\question[1]
下面关于正则表达式BRE,ERE和PCRE的论断,正确的是?
\begin{choices}
	\choice BRE与ERE/PCRE的主要区别是其是否支持部分元字符的特殊意义
	\choice 在类UNIX操作系统中,默认使用的是BRE
	\choice sed和awk只能支持BRE,而grep可支持三种正则表达式
	\choice ERE可支持非贪婪匹配、命名捕获组和零宽断言
\end{choices}
\answerline

\question[1]
下面的正则表达式中,正确的是?
\begin{choices}
	\choice \lstinline|(?<=[stm]*)asd|是一种look-behind
	\choice \lstinline|asd(?=[stm]*)|是一种look-ahead
	\choice	\lstinline|\d+(\.\d{,5})?|表示至多5个小数位的正实数
	\choice \lstinline|[.]*|可以表示任意长度的任意字符
\end{choices}
\answerline

\end{questions}

\section*{二、填空题}

\begin{questions}
	\question[5] 在Linux系统中,有多种特殊文件类型,分别是\fillin[], \fillin[], \fillin[], \fillin[], \fillin[]。
	\question[1] Linux文件系统中每个文件用 \fillin[]来标识。
	\question[2] 链接文件一般分为\fillin[] 和 \fillin[]。
	\question[2] 某文件的权限为:\lstinline{drw-r--r--},用数值形式表示该权限,则该八进制数为: \fillin[],该文件属性是 \fillin[]。
	\question[2] 前台起动的进程使用 \fillin[]终止,使用\fillin[]挂起。
	\question[2] 安装Linux系统对硬盘分区时,必须有两种分区类型: \fillin[] 和\fillin[] 。
	\question[2] 编写的Shell程序运行前必须赋予该脚本文件 \fillin[]和\fillin[]权限。
	\question[4] 系统管理的任务之一是能够实现对程序和数据的\fillin[]、\fillin[]、\fillin[]和\fillin[]。
	\question[2] 系统交换分区(swap)是作为\fillin[]的一块区域。
	\question[4] 内核分为 \fillin[][2in]、 \fillin[][2in]、\fillin[][2in] 和\fillin[][2in] 等四个子系统。
	\question[6] 在安装Linux系统中,需要对网络进行配置,该安装程序会提示用户输入\fillin[]、\fillin[]、\fillin[]、
		\fillin[]、 \fillin[] 和 \fillin[] 等必要信息。
	\question[2] 唯一标识每一个用户的是\fillin[]和\fillin[]。
 	\question[1] \fillin[][2in]可以实现动态 IP地址分配。
	\question[4] 系统网络管理员的管理对象是\fillin[]、 \fillin[] 和\fillin[]以及\fillin[]。
	\question[1] \fillin[]命令可以测试网络中本机系统是否能到达 一台远程主机 ,所以常常用于测试网络的 连通性 。
	\question[2] vim 编辑器具有两种工作模式:\fillin[] 和 \fillin[]。
	\question[1] DNS实际上是分布在internet上的主机信息的数据库,其作用是实现 \fillin[][2in]之间的转换。
	\question[1] 在超级用户下显示Linux系统中正在运行的全部进程,应使用的命令及参数是 \fillin[]。
	\question[1] 在 Linux系统中,压缩文件后生成后缀为.tgz文件的命令是\fillin[]。
	\question[1] 在用vim编辑文件时,将文件内容存入test.txt文件中,应在命令模式下键入 \fillin[]。
\end{questions}

\section*{三、简答题}

\begin{questions}

\question[5]
一般说来,在安装Linux操作系统时,需要对磁盘进行分区,一般的系统分区标准是什么?(假设硬盘为300G,内存16G)
\begin{solution}
\vspace{2in}
\end{solution}
\vspace{2in}

\question[5]
Linux网卡配置文件路径是什么?要使服务器上外网,必须满足的条件有哪些?需要配置什么?
\begin{solution}
\vspace{2in}
\end{solution}
\vspace{2in}

\question[5]
/mnt目录主要用于什么?/root目录跟root用户有什么关系?/根目录与/boot目录有什么联系?
\begin{solution}
\vspace{2in}
\end{solution}
\vspace{2in}

\question[5]
某一天误操作,执行了rm  -rf  * ,会有哪些情况发生?请举例。
\begin{solution}
\vspace{2in}
\end{solution}
\vspace{2in}

\question[5]
在生物序列分析中,可以用正则表达式表示一组序列motif,例如
\begin{lstlisting}
[AC]-x-V-x(4)-{ED}
\end{lstlisting}
这里使用了不同的规则:
\begin{table}[ht]
\begin{tabular}{lll}
\hline
\textbf{Expression} & \textbf{Description} & \textbf{Example}\\ 
\hline
\texttt{x} & ANY single amino acid. & \\
\texttt{[]} & OR & \texttt{[ILV]}: I or L or V\\
\texttt{\{\}} & NOT & \texttt{\{DE\}}: not D or E\\
\texttt{()} & Repetitions & \texttt{x(2,3)}: x-x or x-x-x\\
\texttt{-}	& Separator & \\
\texttt{<} & N-terminal & \\
\texttt{>} & C-terminal & \\
\texttt{.} & END & \\
\hline
\end{tabular}
\end{table}
请写出至少5种可能的多肽序列。

\begin{solution}
\vspace{2in}
\end{solution}



\question[5]
说说基本正则表达式BRE、扩展正则表达式ERE、PCRE各自有什么特性和区别。
\begin{solution}
\vspace{2in}
\end{solution}

\question[5]
下面是常用的国内固定电话号码的表示方式,使用正则表达式捕获下列电话号码中多区号信息:
\begin{lstlisting}
021-3420-4348
(021)3420-4348
02134204348
021 3420 4348
86-21-3420-4348
\end{lstlisting}
\begin{solution}
\vspace{2in}
\end{solution}

\question[5]
写一个正则表达式,匹配下面所有的email,同时用一个组捕获用户名。
\begin{lstlisting}
tom@sina.com
tom.riddle@sina.com
tom.riddle+regexone@sina.com
tom@sina.com.hk
potter@sina.com
harry@sina.com.cn
herminorie+regexone@sina.com
\end{lstlisting}
\begin{solution}
\vspace{2in}
\end{solution}

\question[5]
写出一个正则表达式,匹配下面所有的HTML表达式,同时捕获其中的HTML标签:
\begin{lstlisting}
<a>This is a link</a>
<a href='https://gmail.com'>Gmail</a>
<div class="test_style">Test</div>
<div id="hello">Hello <span>world</span></div>
\end{lstlisting}
\begin{solution}
\vspace{2in}
\end{solution}

\question[5]
写一个正则表达式,匹配下面所有图文件的同时,捕获这些文件的文件名(不包括扩展名)。
\begin{lstlisting}
.bash_profile
cheatsheet.doc
img0330.jpg
updated_img0330.png
documentation.htm
favicon.gif
img0330.jpg.tmp
access.deny
\end{lstlisting}
\begin{solution}
\vspace{2in}
\end{solution}


\question[5]
下面是一些java运行报错的日志信息,请写出一个正则表达式匹配,同时捕获出错的类名以及对应的java文件和行。
\begin{lstlisting}
E/( 1553):   at widget.List.makeView(ListView.java:1727)
E/( 1553):   at widget.List.fillDown(ListView.java:652)
E/( 1553):   at widget.List.fillFrom(ListView.java:709)
\end{lstlisting}
\begin{solution}
\vspace{2in}
\end{solution}


\question[5]
使用正则表达式捕获下列URL的协议(Protocol,如http等)、端口、域名(domain name)等信息:
\begin{lstlisting}
ftp://portal.sjtu.edu.cn:2021/my_secret/life_changing_plans.pdf
https://cbb.sjtu.edu.cn/course/regex/introduction#section
file://localhost:4040/tgz_files
https://s3cur3-server.com:9999/
market://search/angry%20birds
\end{lstlisting}
\begin{solution}
\vspace{2in}
\end{solution}


\end{questions}

\end{CJK*}
\end{document}
