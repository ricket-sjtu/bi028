\documentclass[12pt]{article}

\usepackage{color}
\usepackage{amsmath,amssymb}
\usepackage{enumerate}
\usepackage{CJK}
\usepackage{xcolor}
\usepackage{listings}
\lstset{
language=bash,
keywordstyle=\color{blue!70}\bfseries,
basicstyle=\ttfamily\footnotesize,
commentstyle=\ttfamily\color{red},
showspaces=false,
showstringspaces=false,
showtabs=false,
columns=fullflexible,
frame=no,
rulesepcolor=\color{red!20!green!20!blue!20}
}             
\begin{document}
\begin{CJK*}{UTF8}{gbsn}

\title{上机实验2:挑战正则表达式}
\author{}
\date{}

\maketitle

\section{用户数据库管理}

这里有一个用户数据库文件user,其格式是:
\begin{quote}
用户名:密码:用户标识(UID):组标识(GID):用户描述:主目录:登录Shell
\end{quote}

\begin{enumerate}[(1)]
	\item 分别用grep,sed,awk找出登录shell为bash的用户。
	\item 找出UID或GID是4位以上数字的用户。
	\item 有个主目录在/var/spo??l下的用户,但不能肯定有几个o,找出满足条件的用户。
	\item 已知有个用户名形如"mi...M"中间包含多个未知的mi,找出该用户。
\end{enumerate}

\section{普通文本分析}

这里有个很乱的文本文件regex.txt,需要从中进行一些正则表达式分析:
\begin{enumerate}[(1)]
	\item 用grep查找其中的大写字母
	\item 用grep查找其中的数字
	\item 查找包含"the"的行,忽略大小写,统计次数
	\item 统计不包含"the"的行数
	\item 查找包含"tast"或"test"的行
	\item 查找不包含"\#"的行
	\item 查找"oog"但前面的字符不能为"g"或者"o"
	\item 查找以"the"开头的行
	\item 查找以字母"d"结束的行
	\item 过滤掉空行
	\item 过滤掉注释行,也就是以"\#"开头的行
	\item 查找至少存在两个连续字符"e"的行
	\item 查找字母"g"后面连接2-5个"o"且后面还有一个"g"的的行
	\item 输出regex.txt的内容,同时打印行号,并删除2-5行
	\item 同上,但删除第5行到最后一行
	\item 在源文件中删除第1行
	\item 在第2行前添加两行"test",第3行后添加";;"行
	\item 将2-5行的内容替换为"nothing but 2-5"
	\item 只输出5-8行
\end{enumerate}

\section{获取本机IP地址}

已知命令\lstinline{ifconfig}可以输出本机的网络信息,能否从中提取IP地址?


\section{awk练习}

有个文件competition.txt,里面记录的是红/蓝/绿三队队员的比赛得分情况。

写一个awk脚本:
\begin{enumerate}
	\item 计算每个队员的平均得分
	\item 计算每次比赛的平均得分
	\item 计算每支队伍的平均得分
\end{enumerate}

\textcolor{red}{注意}:
\begin{itemize}
	\item 文件的首行是文件头,处理的时候该注意。
	\item 如果某个得分为负值,说明该队员缺席了该比赛,记分时候应该忽略。
	\item 输出结果时,字符串左对齐排列,占据10个字符;数值按右对齐排列,占据7个字符,小数为2位。
	\item 你的脚本应该不局限于处理本文件。
\end{itemize}


\section{序列分析处理}

这里有一个FASTA文件twoseqs.fa,包含两个序列。用awk写一个脚本
\begin{enumerate}[(1)]
	\item 计算每个序列的3-mer的发生率。所谓3-mer指的是序列中长度为3的子串,如在序列ACTGGACT中ACT的发生率为2。
	\item 根据下式计算两序列的相似性得分:
$$
\textrm{score} = \exp(-\sum_i (\text{occ}_{1i} - \text{occ}_{2i}))
$$
\end{enumerate}


\end{CJK*}
\end{document}
