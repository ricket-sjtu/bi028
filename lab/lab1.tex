\documentclass[12pt]{article}

\usepackage{color}
\usepackage{amsmath,amssymb}
\usepackage{enumerate}
\usepackage{CJK}
\usepackage{xcolor}
\usepackage{listings}
\lstset{
language=bash,
keywordstyle=\color{blue!70}\bfseries,
basicstyle=\ttfamily\footnotesize,
commentstyle=\ttfamily\color{red},
showspaces=false,
showstringspaces=false,
showtabs=false,
columns=fullflexible,
frame=no,
rulesepcolor=\color{red!20!green!20!blue!20},
breaklines=true}


\begin{document}
\begin{CJK*}{UTF8}{gbsn}

\title{实验 1: Linux基本命令行操作}
\author{}
\date{}

\maketitle


\begin{enumerate}[(1)]
	\item 以bio用户登录,在家目录下创建目录\lstinline{~/bi296/lab1commands}。
		以后所有的上机工作都放在\lstinline{~/bi296}下。
	\item 为该目录\lstinline{lab1commands}添加符号连接\lstinline{lab1}
	\item 用\lstinline{tree}命令获取根目录\lstinline{/}下的两层文件目录
		结构,将其重定向到一个文件\lstinline{directory_tree}中。
	\item 用\lstinline{man hier}获取每个目录的基本信息,将其用\lstinline{vim}
		写入文件\lstinline{directory_tree}中
	\item 用\lstinline{man builtin}获取所有bash内置命令的列表,将其写入
		文件\lstinline{builtins},将其修改为每行一个命令的记录。将每个文件的功能
		记入该文件。
	\item 在目录下创建\lstinline{hello.sh}文件,内容为:  
	\begin{lstlisting}
#!/bin/bash
echo "Hello, world"
exit 0
	\end{lstlisting}
	\item 给\lstinline{hello.sh}文件增加执行权限,并删除其他用户的写权限,最终
		以数字写出文件的权限,然后尝试运行该文件。
	\item 写一个C程序,用GCC编译后查看产生的可执行文件,看看其文件类型与此前的可执行文件
		有什么差异:
	\begin{lstlisting}[language=c]
#include <stdio.h>
int main(){
	printf("Hello, World!\n");
	return 0;
}
	\end{lstlisting}
	\item 从搜索引擎获取关于\texttt{bioinformatics}在wikipedia上的页面,将其源码
		用\lstinline{vim}命令编辑为\lstinline{bioinformatics},看看在vim下如何跳转
		到末行,首行,行首、行末,如何在光标行下一行插入,如何复制5行,删除10行,查找
		bioinformatics的字符,把bioinformatics替换为bioinfo science
	\item 将当前\lstinline{bioinformatics}文件所属的用户修改为abc,组为root,写出命令。
	\item 用\lstinline{touch}命令作用于该文件,修改的是文件的哪些时间戳?
	\item 用\lstinline{find}查找linux系统下以log结尾,30天没有修改,大小大于20K同时具
		有写权限的文件并备份到\lstinline{data/backup/}目录下。
	\item 查找文件系统中所有具有setuid或setguid的文件,以及具有sticky-bit的目录,并将
		列表分别输出到一个文件中,并说说这些文件都有什么样的特征。
	\item 查找系统中所有的socket文件,说说这些文进都有什么样的扩展名,其对应的分别是什么?
	\item 在当前目录下创建一个当天的日期作为目录名的文件夹(提示:当前日期表
		示的方法为:\lstinline{date +%Y%m%d})
	\item 如何查看文件内容,命令有哪些?如何查看文件第10行到30行?如何查看文件倒数第4行?
	\item 查看linux服务器IP的命令,同时只显示包含ip所在的行打印出来。
	\item 为计算机添加普通用户test,设定该用户的家目录并将该用户加入root组
	\item 删除用户test,同时删除其目录。
	\item 将文件夹lab1打包为lab1.tar.gz或者lab1.tar.bz2。
\end{enumerate}

到这里实验结束,相信你对命令行已经掌握了大部分了,还有不明白的,记住去问man。

\end{CJK*}
\end{document}
